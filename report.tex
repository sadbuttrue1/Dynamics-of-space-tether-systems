\documentclass[12pt,a4paper]{extarticle}
\usepackage{extsizes}
\usepackage{cmap} % для кодировки шрифтов в pdf
%\documentclass[a4paper,12pt]{article} %размер бумаги устанавливаем А4, шрифт 12пунктов
\usepackage[utf8]{inputenc}%включаем свою кодировку: koi8-r или utf8 в UNIX, cp1251 в Windows
\usepackage[T2A, T1]{fontenc}
\usepackage[english,russian]{babel}%используем русский и английский языки с переносами
\usepackage{amssymb,amsfonts,amsmath,mathtext,cite,enumerate,amsthm,mathenv} %подключаем нужные пакеты расширений
\numberwithin{equation}{section}
\usepackage{graphicx} %хотим вставлять в диплом рисунки?
\usepackage{indentfirst} % Красная строка
\graphicspath{{images/}}%путь к рисункам
\usepackage{hyperref}
\usepackage{framed}
\usepackage{color} %% это для отображения цвета в коде
\usepackage{listings} %% собственно, это и есть пакет listingsb
\usepackage{ucs}
\usepackage{latexsym}
\usepackage{bm}
\usepackage{array}
\usepackage{multirow}
\usepackage{ulem}
\usepackage{csvsimple}

\frenchspacing
%----------------ЗАГОЛОВКИ---------------------------
\usepackage{titlesec}

\titleformat{\chapter}[display]
    {\filcenter}
    {\MakeUppercase{\chaptertitlename} \thechapter}
    {8pt}
    {\bfseries}{}
 
\titleformat{\section}
    {\centering\normalsize\bfseries}
    {\thesection}
    {1em}{\MakeUppercase}
 
\titleformat{\subsection}
    {\normalsize\bfseries}
    {\thesubsection}
    {1em}{}

% Настройка вертикальных и горизонтальных отступов
%\titlespacing*{\chapter}{0pt}{-30pt}{8pt}
\titlespacing*{\section}{\parindent}{*4}{*4}
\titlespacing*{\subsection}{\parindent}{*4}{*4}
%------------------------------------------------------

%---------------------ПОЛЯ-----------------------------
\usepackage{geometry}
\geometry{left=3cm}
\geometry{right=1.5cm}
\geometry{top=2.4cm}
\geometry{bottom=2.4cm}


\makeatletter
\renewcommand{\@biblabel}[1]{#1.} % Заменяем библиографию с квадратных скобок на точку:
\makeatother

\makeatletter


\def\x@multispan#1{%
  \global\let\@tempa\@empty
  \@multicnt#1\relax
  \loop\ifnum\@multicnt>\@ne
  \xdef\@tempa{\@tempa\kern\dimen@i\hfill&\omit}%
   \advance\@multicnt\m@ne
  \repeat
  \@tempa\kern\dimen@i\hfill}


\long\def\xmulticolumn#1#2#3{%
 \omit
 \begingroup
   \def\@addamp{\if@firstamp \@firstampfalse \else
                \@preamerr 5\fi}%
  \@mkpream{#2}\@addtopreamble\@empty
  \endgroup
  \def\@sharp{#3}%
  \setbox\z@\hbox{{\@preamble}}%
\global\dimen@i\wd\z@
\global\divide\dimen@i#1\relax
 \ignorespaces
\x@multispan{#1}}
\makeatother

\linespread{1.3} % полуторный интервал
%\renewcommand{\baselinestretch}{1.5}
\renewcommand{\theenumi}{\arabic{enumi}}% Меняем везде перечисления на цифра.цифра
\renewcommand{\labelenumi}{\arabic{enumi}}% Меняем везде перечисления на цифра.цифра
\renewcommand{\theenumii}{.\arabic{enumii}}% Меняем везде перечисления на цифра.цифра
\renewcommand{\labelenumii}{\arabic{enumi}.\arabic{enumii}.}% Меняем везде перечисления на цифра.цифра
\renewcommand{\theenumiii}{.\arabic{enumiii}}% Меняем везде перечисления на цифра.цифра
\renewcommand{\labelenumiii}{\arabic{enumi}.\arabic{enumii}.\arabic{enumiii}.}% Меняем везде перечисления на цифра.цифра
%\renewcommand{\figurename}{Рисунок} 
\addto\captionsrussian{
\def\figurename{Рисунок}
\renewcommand{\refname}
    {Список использованных источников}
}
\usepackage[labelsep=space]{caption}
\DeclareCaptionLabelSeparator{bar}{ - }
\captionsetup{
  labelsep=bar
}

\usepackage{floatrow}
\DeclareFloatFont{tiny}{\tiny}
\floatsetup[table]{font=footnotesize,capposition=top}

%\usepackage{sectsty}

%\allsectionsfont{\centering}


\lstset{ %
language=Python,                 % выбор языка для подсветки (здесь это Python)
basicstyle=\scriptsize\sffamily, % размер и начертание шрифта для подсветки кода
numberstyle=\tiny,           % размер шрифта для номеров строк
stepnumber=1,                   % размер шага между двумя номерами строк
numbersep=5pt,                % как далеко отстоят номера строк от подсвечиваемого кода
backgroundcolor=\color{white}, % цвет фона подсветки - используем \usepackage{color}
showspaces=false,            % показывать или нет пробелы специальными отступами
showstringspaces=false,      % показывать или нет пробелы в строках
showtabs=false,             % показывать или нет табуляцию в строках
tabsize=2,                 % размер табуляции по умолчанию равен 2 пробелам
captionpos=t,              % позиция заголовка вверху [t] или внизу [b] 
breaklines=true,           % автоматически переносить строки (да\нет)
breakatwhitespace=false, % переносить строки только если есть пробел
escapeinside={\%*}{*)}   % если нужно добавить комментарии в коде
}

%\usepackage[explicit]{titlesec}
%\usepackage{textcase}
%\usepackage{microtype}
%\titleformat{\section}
  %{\normalfont\Large\scshape}{\large\thesection}{1em}{\textls{\MakeTextLowercase{#1}}}
\usepackage{tocloft}
\renewcommand{\cfttoctitlefont}{\hfil \normalfont\Large\bfseries\MakeUppercase}

\usepackage{inconsolata}

\begin{document}
\begin{titlepage}
\begin{center}
\vspace{1.5em}
МИНИСТЕРСТВО ОБРАЗОВАНИЯ И НАУКИ РОССИЙСКОЙ ФЕДЕРАЦИИ\\
\vspace{\baselineskip}
ФЕДЕРАЛЬНОЕ ГОСУДАРСТВЕННОЕ АВТОНОМНОЕ ОБРАЗОВАТЕЛЬНОЕ\\
УЧРЕЖДЕНИЕ ВЫСШЕГО ОБРАЗОВАНИЯ\\
<<САМАРСКИЙ НАЦИОНАЛЬНЫЙ ИССЛЕДОВАТЕЛЬСКИЙ УНИВЕРСИТЕТ\\
ИМЕНИ АКАДЕМИКА С.П. КОРОЛЕВА>>\\
\vspace{\baselineskip}
\end{center}
\begin{center}
  {Институт ракетно-космической техники}\\
  \vspace{\baselineskip}
  {Кафедра теоретической механики}
\end{center}
\vspace{64pt}
\begin{center}
Пояснительная записка к курсовой работе по предмету «Динамика космических тросовых систем»\\
\end{center}
\vspace{64pt}
\begin{minipage}{0.4\linewidth}
  \begin{flushleft}
    {Выполнил студент:} \\
    {Группа:}\\
    {Проверил:}\\
    \hspace{1pt}
    \end{flushleft}
  \end{minipage} 
  \hfill
  \begin{minipage}{0.4\linewidth}
  \begin{flushright}
  Асланов Евгений\\
  1225 М 403\\
  Ледков А. С.\\
  \hspace{64pt}
  \end{flushright}
 \end{minipage}

\vspace{\fill}

\begin{center}
Самара 2017 г.
\end{center}
\end{titlepage}
% это титульный лист
\setcounter{page}{2}
\newpage
\begin{center}
ЗАДАНИЕ\\
на курсовую работу\\
по предмету «Динамика космических тросовых систем»\\
студенту Асланову Е.В. группы 1225 М 403\\
\vspace{32pt}

ЗАДАНИЕ

\begin{enumerate}
	\item Рассчитать высоту орбиты отделения груза для системы состоящей из носителя, троса и груза. Отделение происходит при $\phi=0$ ($phi$ – угол между радиус-вектором носителя и тросом). Начальная длина троса $l_0=2000$м. Скорость разматывания троса:
	\begin{equation*}
		\frac{dl}{dt}=
		\begin{cases}
			5 + k/2, l < l_{max},\\
			0, l \geq l_{max},
		\end{cases}
	\end{equation*}
	где $l_{max} = 30000$м, $k=2$. Высота круговой орбиты носителя $H=250$км. Коэффициент упругости троса $c=0$. Масса носителя $m_1=6000$кг, масса груза $m_2=50$кг.
	\item Рассчитать угол отклонения космического лифта при подъеме груза. Масса троса $m_c = 5000$т, масса подъемника $m_e=100$кг. Коэффициент упругости троса $c=0$. Длина троса $h=114000000$м. Скорость подъемника $V_e=100$м/с.
\end{enumerate}
\end{center}

\newpage
\tableofcontents % это оглавление, которое генерируется автоматически
\newpage
\section*{ВВЕДЕНИЕ}
\addcontentsline{toc}{section}{Введение}

Данная работа является учебной и её цель – закрепление пройденного курса динамика космических тросовых систем.
\newpage
\section{Исследование движения спускаемого груза}

Определим радиус-векторы носителя и груза:
\begin{equation*}
	r_1 = (r * \cos \theta, r * \sin \theta),
\end{equation*}
\begin{equation*}
	r_2 = r_1 + (l*\sin \alpha, - l * \cos \alpha),
\end{equation*},
где $\alpha=\phi + \theta - \frac{\pi}{2}$.

Запишем кинетическую и потенциальную энергии для получения лагранжиана $L = T - \Pi$:
\begin{equation*}
	T = \frac{m_1 \mathbf{v_1}^2}{2} + \frac{m_2 \mathbf{v_2}^2}{2},
\end{equation*}
\begin{equation*}
	\Pi = \frac{-m_1 \mu}{\mathbf{r_1}} + \frac{-m_2 \mu}{\mathbf{r_2}} + \frac{c}{2 (l-l_0)^2}.
\end{equation*}

Решив уравнение Лагранжа второго рода для $\phi$ и $\frac{d\phi}{dt}$, получим следующие зависимости:
\begin{figure}[H]
	\center{\includegraphics[scale=0.6]{task1/phi.png}}
	\caption{Изменение угла троса}
\end{figure}
\begin{figure}[H]
	\center{\includegraphics[scale=0.6]{task1/dphi.png}}
	\caption{Изменение скорости угла троса}
\end{figure}

Для полученных зависимостей перигей орбиты отделения $r_p = 79595.0335$м.
\newpage
\input{task2}
\newpage
\section*{ЗАКЛЮЧЕНИЕ}
\addcontentsline{toc}{section}{Заключение}

В первой части работы было смоделировано поведение троса и рассчитана орбита груза после отделения.

Во второй части работы было смоделировано поведение космического лифта и рассчитано время подъема.

В приложении приведена программа на языке программирования Python с использованием библиотек numpy, scipy, sympy, matplotlib.
\newpage
\addcontentsline{toc}{section}{Список использованных источников\bibliographystyle{gost780s} }
\begin{thebibliography}{99}
\bibitem{} Арцутанов Ю.Н. В космос без ракет // Знание - сила. 1969. No 7. С. 25.
\bibitem{} Маркеев А.П. Теоретическая механика. М.: ЧеРо, 1999. 
\bibitem{} Поляков Г.Г. Привязные спутники, космические лифты и кольца. Астрахань: Изд-во
Астраханского гос. пед. ун-та, 1999. 
\end{thebibliography}
\newpage
\section*{ПРИЛОЖЕНИЕ А\break Исходный текст программы для численного решения задач курсовой работы}\addcontentsline{toc}{section}{Приложение А Исходный текст программы для численного решения\break задач курсовой работы}
\lstinputlisting{task1.py}
\lstinputlisting{task2.py}
\end{document}