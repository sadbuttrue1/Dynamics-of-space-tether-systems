\begin{center}
ЗАДАНИЕ\\
на курсовую работу\\
по предмету «Динамика космических тросовых систем»\\
студенту Асланову Е.В. группы 1225 М 403\\
\vspace{32pt}

ЗАДАНИЕ

\begin{enumerate}
	\item Рассчитать высоту орбиты отделения груза для системы состоящей из носителя, троса и груза. Отделение происходит при $\phi=0$ ($phi$ – угол между радиус-вектором носителя и тросом). Начальная длина троса $l_0=2000$м. Скорость разматывания троса:
	\begin{equation*}
		\frac{dl}{dt}=
		\begin{cases}
			5 + k/2, l < l_{max},\\
			0, l \geq l_{max},
		\end{cases}
	\end{equation*}
	где $l_{max} = 30000$м, $k=2$. Высота круговой орбиты носителя $H=250$км. Коэффициент упругости троса $c=0$. Масса носителя $m_1=6000$кг, масса груза $m_2=50$кг.
\end{enumerate}
\end{center}
